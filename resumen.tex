\chapter[Resumen]{Resumen y Organización de la memoria}

La memoria presentada a continuación se divide en cinco partes.

En la primera parte, denominada \textit{Alcance del Trabajo}, se presenta un resumen sobre en que consiste el proyecto realizado, enumerando objetivos y aportando una primera visión inicial.

En la segunda parte, \textit{Estado del Arte de la Seguridad Informática}, se muestra uno de los dos bloques de los que consta el proyecto. Un estado del arte exhaustivo sobre la Seguridad Informática, sus aplicaciones y el Pentesting.

En la tercera parte, \textit{Desarrollo de la Aplicación}, se muestra el desarrollo del segundo bloque del proyecto, que reside en una aplicación que implementa una solución a un problema concreto, basándose en todo lo aprendido en el primer bloque.

En la cuarta parte, \textit{Análisis y conclusiones del Trabajo}, se recopilan las conclusiones extraídas tras desarrollar todo el proyecto.

En la ultima, \textit{Apéndices}, se incluye la bibliografía utilizada y referencias al contenido desarrollado