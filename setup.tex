% IMPORTANTE
% Para poner barrabaja en una url hay que cambiar '_' por '\textunderscore'

%==============================================================================
% PACKAGES
%==============================================================================

\usepackage[spanish,es-tabla]{babel} % For spanish texts
\usepackage[utf8]{inputenc} % To use accent marks directly
\usepackage[T1]{fontenc}
\usepackage{graphicx} % For graphics
\usepackage{svg} % For vectorial graphics
\usepackage{float} % To modify the pisotion of pictures
\usepackage{listings} % For code
\usepackage{times} % For dates
\usepackage{xcolor} % For colors (better than 'color' package)
\usepackage[pdftex,
			pdfauthor={Ander Granado Masid},
			pdftitle={},
			pdfsubject={},
			pdfkeywords={},
			pdfproducer={Latex},
			pdfcreator={pdflatex}]{hyperref}  % For links and  PDF metadata
\usepackage{url}
\usepackage[Bjornstrup]{fncychap} % For better chapter titles
\usepackage{fancyhdr} % For pages header and footer
\usepackage{titling} % For use doc info in other parts of it 
\usepackage{caption} % To add campions to images or tables
\usepackage{minted} % Better syntax highlighting that listings
\usepackage{comment} % To make multiple line comments
\usepackage{enumitem} % For add custom enumeration items
\usepackage{framed} % For frames for the tasks
\usepackage{lscape} % To make some pages in landscape in a portrait document
\usepackage{eurosym} % For euro symbol
\usepackage{epigraph} % For epigraphs at the start of the chapters
\usepackage{tablefootnote} % To make LaTeX able to put a footnote referenced inside a table
\usepackage{todonotes}
\usepackage[backend=biber,style=numeric,sorting=none]{biblatex} % For references
\addbibresource{biblio.bib} 

%==============================================================================
% REDEFINITIONS THAT I USE BUT IDK WTF THEY ARE DOING (CARE WITH THIS!)
%==============================================================================

%----------------------------------------------------------------------------------------
% REDEF 1: FOR PART TITLES
% (http://tex.stackexchange.com/questions/36902/questions-about-bjornstrup)
%----------------------------------------------------------------------------------------

\colorlet{partbgcolor}{gray!30}% shaded background color for parts
\colorlet{partnumcolor}{gray}% color for numbers in parts
\colorlet{chapbgcolor}{gray!20}% shaded background color for chapters
\colorlet{chapnumcolor}{gray}% color for numbers in chapters

\renewcommand*\partformat{%
  \fontsize{76}{80}\usefont{T1}{pzc}{m}{n}\selectfont%
  \hfill\textcolor{partnumcolor}{\thepart}}

\makeatletter
\renewcommand*{\@part}{}
\def\@part[#1]#2{%
  \ifnum \c@secnumdepth >-2\relax
    \refstepcounter{part}%
    \@maybeautodot\thepart%
    \addparttocentry{\thepart}{#1}%
  \else
    \addparttocentry{}{#1}%
  \fi
  \begingroup
    \setparsizes{\z@}{\z@}{\z@\@plus 1fil}\par@updaterelative
    \raggedpart
    \interlinepenalty \@M
    \normalfont\sectfont\nobreak
    \setlength\fboxsep{0pt}
    \colorbox{partbgcolor}{\rule{0pt}{40pt}%
    \makebox[\linewidth]{%
    \begin{minipage}{\dimexpr\linewidth+20pt\relax}
      \ifnum \c@secnumdepth >-2\relax
        \vskip-25pt
        \size@partnumber{\partformat}%
      \fi      %
      \vskip\baselineskip
      \hspace*{\dimexpr\myhi+10pt\relax}%
      \parbox{\dimexpr\linewidth-2\myhi-20pt\relax}{\raggedleft\LARGE#2\strut}%
      \hspace*{\myhi}\par\medskip%
    \end{minipage}%
      }%
    }%
    \partmark{#1}\par
  \endgroup
  \@endpart
}

\renewcommand\DOCH{%
  \settowidth{\py}{\CNoV\thechapter}
  \addtolength{\py}{-10pt}
  \fboxsep=0pt%
  \colorbox{chapbgcolor}{\rule{0pt}{40pt}\parbox[b]{\textwidth}{\hfill}}%
  \kern-\py\raise20pt%
  \hbox{\color{chapnumcolor}\CNoV\thechapter}\\%
}

\renewcommand\DOTI[1]{%
  \nointerlineskip\raggedright%
  \fboxsep=\myhi%
  \vskip-1ex%
  \colorbox{chapbgcolor}{\parbox[t]{\mylen}{\CTV\FmTi{#1}}}\par\nobreak%
  \vskip 40pt%
}

\renewcommand\DOTIS[1]{%
  \fboxsep=0pt
  \colorbox{chapbgcolor}{\rule{0pt}{40pt}\parbox[b]{\textwidth}{\hfill}}\\%
  \nointerlineskip\raggedright%
  \fboxsep=\myhi%
  \colorbox{chapbgcolor}{\parbox[t]{\mylen}{\CTV\FmTi{#1}}}\par\nobreak%
  \vskip 40pt%
 }
\makeatother

%==============================================================================
% STYLE
%==============================================================================

%------------------------------------------------------------------------------
% PATH FOR THE PICTURES
%------------------------------------------------------------------------------

\graphicspath{{./figuras/}}

%------------------------------------------------------------------------------
% FIX FOR THE LIST OF FIGURES
%------------------------------------------------------------------------------

\makeatletter
\renewcommand*\l@figure{\@dottedtocline{1}{1em}{3.2em}}
\makeatother

%------------------------------------------------------------------------------
% TYPOGRAPHY
%------------------------------------------------------------------------------

\renewcommand{\familydefault}{\rmdefault} % Canges typography for ALL document
\setlength{\parskip}{1mm} % Defines the lenght between paragraphs (def == 0)
\renewcommand{\baselinestretch}{1.0}

%------------------------------------------------------------------------------
% COLORS FOR PART AND CHAPTER TITLES
%------------------------------------------------------------------------------

\definecolor{TitlesLight}{RGB}{112, 183, 172}
\definecolor{TitlesDark}{RGB}{28, 85, 76}

\colorlet{partbgcolor}{TitlesLight}% shaded background color for parts
\colorlet{partnumcolor}{TitlesDark}% color for numbers in parts

%------------------------------------------------------------------------------
% HEADERS AND FOOTERS
%------------------------------------------------------------------------------

\pagestyle{fancy}
\fancyhf{}
\fancyhead[RE,LO]{{\small \leftmark}}
\fancyhead[LE,RO]{\thepage}
\fancyfoot[RE,LO]{\theauthor}
\fancyfoot[LE,RO]{\thepage}

\renewcommand{\headrulewidth}{1pt}
\renewcommand{\footrulewidth}{1pt}

%------------------------------------------------------------------------------
% EPIGRAPHS
%------------------------------------------------------------------------------

\setlength\epigraphwidth{.6\textwidth}
\setlength\epigraphrule{0pt}

%------------------------------------------------------------------------------
% LINKS (All black)
%------------------------------------------------------------------------------

\hypersetup{
	colorlinks,
	citecolor=black,
	filecolor=black,
	linkcolor=black,
	urlcolor=black
}

%------------------------------------------------------------------------------
% CONTENTS NUMBERING 
%------------------------------------------------------------------------------

\setcounter{tocdepth}{4}
\setcounter{secnumdepth}{3}

\renewcommand\thechapter{\arabic{chapter}}

%------------------------------------------------------------------------------
% ORDERED LISTS NUMBERING
%------------------------------------------------------------------------------

\newlist{numbered}{enumerate}{4}
\setlist[numbered,1]{label=\textbf{\arabic*.}}
\setlist[numbered,2]{label=\textbf{\arabic{numberedi}.\arabic*}}
\setlist[numbered,3]{label=\textbf{\arabic{numberedi}.\arabic{numberedii}.\arabic*}}
\setlist[numbered,4]{label=\textbf{\arabic{numberedi}.\arabic{numberedii}.\arabic{numberediii}.\arabic*}}

%------------------------------------------------------------------------------
% FRAMES FOR TASKS
%------------------------------------------------------------------------------

\newcommand{\taskframe}[4]{
	\begin{framed}
		\textbf{Número:} #1.
		
		\textbf{Nombre:} #2.
		
		\textbf{Descripción:} #3.
		
		\textbf{Trabajo estimado:} #4 {horas.}
	\end{framed}	
}

\newcommand{\riskframe}[6]{
	\begin{framed}
		\begin{framed}
			\begin{center}
				\textsc{\textbf{#1}}	
			\end{center}
		\end{framed}
		
		\textbf{Descripción:} #2.
		
		\textbf{Probabilidad:} #3.
		
		\textbf{Peligrosidad:} #4.
		
		\textbf{Medidas preventivas:} #5.
			
		\textbf{Medidas correctoras:} #6.
	\end{framed}	
}

%------------------------------------------------------------------------------
% CODE
%------------------------------------------------------------------------------

% Color for commands (sh, bash, ...) (with lstlisting)
\definecolor{gray95}{gray}{.95}
\definecolor{gray85}{gray}{.80}
\definecolor{gray45}{gray}{.45}
\definecolor{myturquoise}{RGB}{0, 128, 128}
\definecolor{mypink}{RGB}{177,48,112}
\definecolor{myblue}{RGB}{56,133,231}

\lstset{ 
	frame=Ltb,
	framerule=0pt,
	aboveskip=0.5cm,
	framextopmargin=3pt,
	framexbottommargin=3pt,
	framexleftmargin=0.2cm,
	framesep=0pt,
	rulesep=2.0pt,
	backgroundcolor=\color{gray95},
	rulesepcolor=\color{black},
	%
	stringstyle=\ttfamily,
	showstringspaces = false,
	basicstyle=\small\ttfamily,
	commentstyle=\color{gray45},
	keywordstyle=\bfseries,
	%
	numbers=left,
	numbersep=15pt,
	numberstyle=\tiny,
	numberfirstline = false,
	breaklines=true,
}

% Minimize listings fragmentation
%\lstnewenvironment{listing}[1][]
%{\lstset{#1}\pagebreak[0]}{\pagebreak[0]}

% Code style for terminal
\lstdefinelanguage{none}{}
\lstdefinestyle{terminal}{
	language=none,
	breaklines=true,
	basicstyle=\footnotesize\bf\ttfamily,
	backgroundcolor=\color{gray85},
	rulesepcolor=\color{myturquoise},
	numbers=none,
}

% Defnition for Kotlin language (because it not exists)
% https://github.com/cansik/kotlin-latex-listing	(thanks mate!)
\lstdefinelanguage{kotlin}{
	keywords={package, as, as?, typealias, this, super, val, var, fun, for, null, true, false, is, in, throw, return, break, continue, object, if, try, else, while, do, when, class, interface, enum, object, companion, override, public, private, get, set, import, abstract, vararg, expect, actual, where, suspend, data, internal, dynamic, final, by},
	keywordstyle=\color{NavyBlue}\bfseries,
	ndkeywords={@Deprecated, @JvmName, @JvmStatic, @JvmOverloads, @JvmField, @JvmSynthetic, Iterable, Int, Long, Integer, Short, Byte, Float, Double, String, Runnable, Array},
	ndkeywordstyle=\color{BurntOrange}\bfseries,
	emph={println, return@, forEach, map, mapNotNull, first, filter, firstOrNull, lazy, delegate},
	emphstyle={\color{OrangeRed}},
	identifierstyle=\color{black},
	sensitive=true,
	commentstyle=\color{gray}\ttfamily,
	comment=[l]{//},
	morecomment=[s]{/*}{*/},
	stringstyle=\color{ForestGreen}\ttfamily,
	morestring=[b]",
	morestring=[s]{"""*}{*"""},
}

% For the rest of the code (with minted)
% (minted have better syntax highlighting than lstlisting for code)
\usemintedstyle{vs}

\setmintedinline{breakanywhere}

\renewcommand\listoflistingscaption{Índice de códigos fuente}