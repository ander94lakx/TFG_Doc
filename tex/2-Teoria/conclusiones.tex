\chapter{Conclusiones}

A lo largo de los capítulos anteriores se ha realizado un profundo análisis sobre el campo de la seguridad informática, explicando una gran cantidad de conceptos. Primero se han definido una serie de conceptos básicos sobre seguridad informática, intrínsecos a cualquier especialidad o aplicación, como son los servicios de la seguridad de la información.

Por otra parte se ha elaborado un estado del arte de las diferentes aplicaciones de la seguridad informática, comenzando desde el surgimiento de la necesidad de proteger la información y de hacer que las transmisiones de datos fuesen seguras. Se hace un énfasis en el malware y en los tipos que existen, mostrando de que manera somos vulnerables a ataques y en que medida nuestra información está desprotegida. 

Además, se desmonta el concepto tan arraigado en las mentes de los usuarios de que la seguridad informática consiste en proteger ordenadores personales de malware. Por una parte el malware no es la única amenaza existente, las intrusiones a redes o lo ataques de ingeniería social son otras formas mediante las que se puede acceder a nuestra información. También hay que tener en cuenta que no solo los ordenadores personales son el único target, los avances en diferentes campos, destacando los smartphones, el Internet of Things y el Cloud Computing, hace que estos sean actualmente vértices de enfoque importantes dentro de la seguridad informática.

Tras ese análisis sobre el mundo de la seguridad informática, conocer los diferentes sistemas y sus medidas de seguridad, mencionar diferentes tecnologías y enumerar y describir diferentes tipos de vulnerabilidades se puede obtener una idea clara sobre la seguridad informática. Aun así, esta idea no estará completa si no se conoce como actuar ante estas amenazas, como un hacker actúa para conseguir información de un sistema. Por ello se ha hecho especial hincapié en el pentesting. 

El pentesting es una técnica que, mediante la imitación del comportamiento de un atacante, permite detectar y explotar vulnerabilidades. Dista de un atacante malicioso, o \textit{Black Hat Hacker}, en que el pentester usa el hacking ético y una serie de procedimientos definidos y controlados con el objetivo de detectar vulnerabilidades de todo tipo para poder posteriormente corregirlas, haciendo los sistemas más seguros. Esta técnica, que requiere un profundo conocimiento de los vectores de ataque y vulnerabilidades, además de sólidas nociones sobre redes y sistemas, permite comprender mejor ya no solo como defenderse sino también como atacar, dando una visión más global del conjunto.

\section{La seguridad informática y los usuarios}

Ante esto se plantea un problema serio con respecto a la seguridad de la información. El estado del arte elaborado contiene una gran cantidad de información sobre la seguridad informática, pero no es algo que un neófito en la materia pueda comprender. La seguridad informática es un campo técnico que requiere de una gran cantidad de conocimientos para poder aplicar sus medidas. Por lo tanto, ¿la seguridad de la información de los usuarios solo puede mantenerse mediante el trabajo de los profesionales de la seguridad? ¿Los usuarios no pueden de manera proactiva controlar sus sistemas haciendo que sean estos más seguros?

Aquí es donde entra en juego la programación. ¿Cómo? Elaborando software que pueda usar cualquier usuario, implementando funcionalidad que permita hacer lo que un hacker realiza en su trabajo, pero de una manera clara y sencilla para el usuario, alejado de aspectos técnicos. Esa es la clave para hacer que los usuarios sean un elemento activo en pro de la seguridad de la información.

Pongamos un ejemplo práctico. Si uno se pone en la piel de un pentester, y quiere hacer un ataque de penetración en una red concreta, buscará vulnerabilidades para poder acceder. También puede querer encontrar maneras para defenderse o detectar elementos en una red. El pentester, un hacker con profundos conocimientos del área, sabe que uan de las mejores herramientas de escaneo de redes es Nmap\footnote{https://nmap.org/}. Mediante un par de comandos en su terminal obtendrá la información que necesita. De esa manera tan simple podrá proseguir con su trabajo.

Ahora pongamos otro caso. Si, en este caso, uno se pone en la piel de un usuario común que simplemente quiere evitar intrusos en su red local. Algo que seria trivial para un hacker o alguien con conocimientos de redes, resulta bastante complicado para un usuario normal. Dicho usuario no conoce ni Nmap ni otras herramientas, no sabe ni siquiera acceder a su router para configurarlo, mucho menos sabe sobre redes o sistemas. Entonces, ¿cómo podría saber si tiene intrusos en su red? ¿La única opción que tendría sería recurrir a alguien con conocimientos sobre el tema?

\section{Herramientas de seguridad informática para usuarios}

En ese tipo de casos es donde entra a colación el software, y como este puede ayudar a los usuarios a llevar a cabo este tipo de tareas. Si queremos ofrecer una solución para el usuario, en base a lo aprendido en puntos anteriores podemos obtener varias conclusiones. 

Lo primero es que, debido al auge de los smartphones, nuestro software debería programarse para uno de estos sistemas. Si debe ir para uno de estos sistemas, y teniendo que elegir uno, lo más lógico sería elegir Android, por las cuotas de mercado mencionadas anteriormente. 

Por otra parte, nuestra aplicación se va a basar en la recogida de información, proceso al que en un pentesting llamábamos \textit{Information Gathering}, por lo que conocer como un pentester recoge la información es fundamental para desarrollar dicha aplicación. Después se puede entrar en que herramientas concretas se puede usar. Nmap parece a primera vista la opción más viable dentro de ese campo. También hay que tener en cuenta herramientas de desarrollo concretas o herramientas secundarias que también pueden ser útiles.

En la siguiente parte de esta memoria se desarrollará una aplicación con lo mencionado en los párrafos anteriores (y algunas herramientas más). Se explicaran en detalles las tecnologías y herramientas usadas y todo el proceso de desarrollo de la aplicación, tanto de programación como de diseño de la interfaz o de la experiencia de usuario, con el objetivo de implementar una prueba de concepto que permita vislumbrar cómo llevar herramientas de seguridad informática a usuarios comunes.