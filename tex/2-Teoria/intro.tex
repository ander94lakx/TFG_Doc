\chapter{Introducción}

La siguiente parte de esta memoria contiene el trabajo realizado durante la Fase 1 del proyecto. En él se elabora un completo estado del arte sobre el mundo de la seguridad informática. Dicho estado del arte se encuentra dividido en cuatro capítulos.

En el primero, simplemente se pasa a mencionar una serie de conceptos básicos sobre seguridad informática, definiciones que es necesario comprender para poder entender la raíz de las seguridad informática.

En el segundo capítulo, llamado \textit{Aplicaciones de la Seguridad Informática}, se enumeran una serie de amenazas hacia la seguridad, pasando de un enfoque más clásico a profundizar en las áreas con más auge dentro del campo, mostrando sus particularidades, con el objetivo de hacerse una visión global de las diferentes necesidades y de cómo la seguridad informática se aplica en cada campo.

En el tercer capítulo, denominado \textit{Pentesting}, se profundiza en la técnica que su propio nombre indica. Dicha técnica es una de las herramientas fundamentales de la seguridad informática, y permite obtener una mejor visión de como securizar sistemas.

Aun así, el objetivo de este estado del arte no es simplemente obtener una visión global del campo de la seguridad informática, sino también ser capaces de extraer una serie de conclusiones con el objetivo final de ofrecer una solución que consiga hacer que el propio usuario, inexperto, pero el principal target de cualquiera de las tecnologías de la información, tenga un papel proactivo en la seguridad de sus sistemas y su información.

De esta manera, de una visión vaga y difuminada de un campo con cada vez mayor importancia, se obtendrá una imagen nítida de todo el área, que además permita enfocar la solución a desarrollar.