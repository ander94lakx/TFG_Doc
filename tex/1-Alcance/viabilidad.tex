\chapter{Viabilidad}
{\color{red} Explicar para que sirve este punto y lo mas importante, explicar la separación en las 2 fases y el por qué de ésta.}

\section{Requisitos funcionales del trabajo}
{\color{red} Poner los RF. Puntos cortos. Nada de párrafos. Como mucho 3 lineas por RF. No tienen que ser específicos, pero tiene que ser cosas que se cumplan si o si. Se pueden hacer a nivel general, si distinguir entre fases.}

%------------------------------------------------------------------------------

\section{Planificación del tiempo}

\subsection[EDT]{Estructura de Descomposición del Trabajo}
{\color{red} El EDT. Se pone la separación en 2 fases, y en cada fases se puede separa por diferentes secciones. Poner por separado el EDT de la fase 1 y el de la Fase 2.}

\subsubsection{Fase 1}

\subsubsection{Fase 2}

\subsection{Agenda del proyecto}
{\color{red} Poner el horario de trabajo, festivos y todo este tema que sirva para elaborar los cronogramas y determinar el tiempo del proyecto.}

%------------------------------------------------------------------------------

\section{Fase 1}
{\color{red} Parte de teoría}

\subsection{Tareas}
{\color{red} Todas las tareas de la Fase 1, primero en lista y después como en cuadritos con descripción y tal.}

\subsection{Entregables}
{\color{red} Entregables durante la Fase 1, si hay (en este caso seria solo la parte de la memoria).}

\subsection{Cronograma}
{\color{red} Distribución de tareas de la Fase 1 en un cronograma generado con Project o una herramienta similar.}

%------------------------------------------------------------------------------

\section{Fase 2}
{\color{red} Parte de la aplicación.}

\subsection{Tareas}
{\color{red} Todas las tareas de la Fase 2, primero en lista y después como en cuadritos con descripción y tal.}

\subsection{Entregables}
{\color{red} Entregables durante la Fase 2, si hay (aplicación + la otra parte de la memoria que sirve como documentación de la aplicación y así).}

\subsection{Cronograma}
{\color{red} Distribución de tareas de la Fase 2 en un cronograma generado con Project o una herramienta similar.}

%------------------------------------------------------------------------------

\section{Gestión de costos}
{\color{red} Poner los recursos que se necesitan.}

\subsection{Presupuesto}
{\color{red} Todas las tablas y así, todo lo dado en la asignatura de Mari Carmen.}

%------------------------------------------------------------------------------

\section{Gestión de riesgos}

\subsection{Fase 1}
{\color{red} Poner los diferentes riesgos de la Fase 1. En esta fase son pocos o ninguno, por el carácter teórico de la fase.}

\subsubsection{Explicación y plan de contingencia}
{\color{red} Explicar dichos riesgos. Poner por puntos. Puntos como descripción, peligrosidad, y medidas preventivas o para corregir. Nada de párrafos.}

\subsection{Fase 2}
{\color{red} Poner los diferentes riesgos de la Fase 2. En esta fase los risegos son mucho mayores, ya que puede haber desde complicaciones a la hora de programar, limitaciones por no ser root,... Mil historias.}

\subsubsection{Explicación y plan de contingencia}
{\color{red} Explicar dichos riesgos. Poner por puntos. Puntos como descripción, peligrosidad, y medidas preventivas o para corregir. Nada de parrafos.}

