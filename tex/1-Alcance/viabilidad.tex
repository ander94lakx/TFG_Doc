\chapter{Viabilidad}
{\color{red} Explicar para que sirve este punto y lo mas importante, explicar la separación en las 2 fases y el por qué de ésta.}

\section{Requisitos funcionales del trabajo}
{\color{red} Poner los RF. Puntos cortos. Nada de párrafos. Como mucho 3 lineas por RF. No tienen que ser específicos, pero tiene que ser cosas que se cumplan si o si. Se pueden hacer a nivel general, si distinguir entre fases.}

%------------------------------------------------------------------------------

\section{Planificación del tiempo}

\subsection[EDT]{Estructura de Descomposición del Trabajo}
{\color{red} El EDT. Se pone la separación en 2 fases, y en cada fases se puede separa por diferentes secciones. Poner por separado el EDT de la fase 1 y el de la Fase 2.}

\subsubsection{Fase 1}

\subsubsection{Fase 2}

\subsection{Agenda del proyecto}
{\color{red} Poner el horario de trabajo, festivos y todo este tema que sirva para elaborar los cronogramas y determinar el tiempo del proyecto.}

\subsection{Tareas}
{\color{red} Todas las tareas tanto de la Fase 1 como de la Fase 2, primero en lista y después como en cuadritos con descripción y tal.}

\begin{numbered}
	\setcounter{numberedi}{-1} %Para empeza a contar desde 0
	
	\item Análisis del proyecto
	\begin{numbered}
		\item Objetivos del proyecto
		\item RF del proyecto
	\end{numbered}
	
	\item FASE 1
	\begin{numbered}
		
		\item Formación
		\begin{numbered}
			\item Formación seguridad informatica
			\item Formación Pentesting
			\begin{numbered}
				\item Formación Teórica
				\item Formación NMap
			\end{numbered}
			\item Formación Android
			\begin{numbered}
				\item Formación Diseño de GUI
				\item Fornacion Java, Gradle, ...
			\end{numbered}
			\item Formación Git
			\item Formacion Latex
		\end{numbered}
		
		\item Elaboración del estado del arte
		\begin{numbered}
			\item Evaluar estado de la profesión
			\item Evaluar diferentes areas y aplicaciones
			\item Documentar estado del arte
		\end{numbered}
		
		\item Preparación de la Fase 2
		\begin{numbered}
			\item Conclusiones de la Fase 1
			\item Revisión de objetivos
			\item Modificación del análisis del proyecto
		\end{numbered}
	\end{numbered}
	
	\item FASE 2
	\begin{numbered}
		\item Preparacion del entorno de trabajo
		\begin{numbered}
			\item Instalación de herramientas
			\item Configuracion de herramientas
		\end{numbered}
		
		\item Desarrollo de la aplicación
		\begin{numbered}
			\item Diseño de la GUI
			\item Implementacion de la apicación
			\begin{numbered}
				\item Programación de herramientas básicas
				\item Integración de Nmap
				\item Programación de parser de elementos
				\item Enlazado de datos con la GUI
			\end{numbered}
			\item Testeo
			\item Corrección de bugs/errores
		\end{numbered}
	\end{numbered}
	
	\item Elaboración de la memoria
	\begin{numbered}
		\item Integración del estado del arte
		\item Documentación de la aplicación desarrollada
		\item Resumen de la memoria
		\item Elaborar presentación
		\item Preparación de la defensa
	\end{numbered}
	
	\item Reuniones periódicas
\end{numbered}

A continuación se explica mediante una breve definición en que consiste cada tarea, ademas de especificar su duración en horas.

\taskframe
	{0.1}
	{Objetivos del proyecto}
	{Definir los objetivos que tiene que cumplir el TFG}
	{5}
\taskframe
	{0.2}
	{RF del proyecto}
	{Definir, en base a los objetivos del proyecto, los Requisitos funcionales concretos del proyecto}
	{5}
\taskframe
	{1.1.1}
	{Formación seguridad informática}
	{Familiarizarse con el amplio entorno de la seguridad informática y comprender las diferentes áreas, objetivos y el estado de dicho campo}
	{30}
\taskframe
	{1.1.2.1}
	{Formación Teórica}
	{Familiarizarse con los conceptos de Pentesting, las diferentes técnicas usadas y las diferentes fases del proceso de Pentesting}
	{20}
\taskframe
	{1.1.2.2}
	{Formación NMap}
	{Familiarizarse con el entorno de NMap, como implementarlo, usarlo para obtener información y de que formas se puede obtener información estructurada y organizada para su posterior uso}
	{10}
\taskframe
	{1.1.3.1}
	{Formación Diseño de GUI}
	{Aprender a usar herramientas de diseño de GUI, diferentes patrones de Diseño en sistemas Android, y el uso de IDEs o herramientas para desarrollar dichas GUIs}
	{10}
\taskframe
	{1.1.3.2}
	{Formación Java, Gradle, ...}
	{Aprender sobre el uso de Java para desarrollar aplicaciones Android, diferentes clases, utilidades o conceptos recurrentes en la programación para Android}
	{15}
\taskframe
	{1.1.4}
	{Formación Git}
	{Aprender el uso de dicho sistema de control de versiones para llevar un control riguroso del desarrollo del proyecto y de la aplicación}
	{5}
\taskframe
	{1.1.5}
	{Formación Latex}
	{Aprender diferentes conceptos de \LaTeX para elaborar tanto el estado del arte como el propio informe de la manera mas clara y elegante posible}
	{5}
\taskframe
	{1.2.1}
	{Evaluar estado de la profesión}
	{Analizar los diferentes campos de la profesión, las necesidades mas demandadas y los diferentes perfiles de profesionales dentro del campo}
	{5}
\taskframe
	{1.2.2}
	{Evaluar diferentes áreas y aplicaciones}
	{Evaluar las necesidades concretas a nivel técnico, las aplicaciones mas usadas y las virtudes y carencias de éstas}
	{10}
\taskframe
	{1.2.3}
	{Documentar estado del arte}
	{Elaborar la documentación en base a toda la información recogida para obtener un elaborado estado del arte}
	{5}
\taskframe
	{1.3.1}
	{Conclusiones de la Fase 1}
	{Elaborar una serie de conclusiones en función a todo el estudio realizado sobre el campo de la seguridad informática}
	{5}
\taskframe
	{1.3.2}
	{Revisión de objetivos}
	{Revisión de los objetivos y los requisitos funcionales de la aplicación a desarrollar en función a todo lo investigado}
	{5}
\taskframe
	{1.3.3}
	{Modificación del análisis del proyecto}
	{Modificar la parte de análisis del proyecto realizada anteriormente, antes de comenzar con la Fase 1}
	{5}					
\taskframe
	{2.1.1}
	{Instalación de herramientas}
	{Instalación de todo lo necesario para desarrollar la aplicación}
	{5}
\taskframe
	{2.1.2}
	{Configuración de herramientas}
	{Configuración de todas las herramientas para que el desarrollo de la aplicación sea lo mas cómodo posible}
	{5}
\taskframe
	{2.2.1}
	{Diseño de la GUI}
	{Diseñar una interfaz gráfica clara y sencilla de usar para interactuar con las funciones a implementar}
	{15}
\taskframe
	{2.2.2.1}
	{Programación de herramientas básicas}
	{Programar herramientas básicas para el escaneo de redes}
	{10}
\taskframe
	{2.2.2.2}
	{Integración de Nmap}
	{Integrar el núcleo de NMap en la aplicación para poder hacer uso de toda su funcionalidad}
	{10}
\taskframe
	{2.2.2.3}
	{Programación de parser de elementos}
	{Elaborar un puente entre NMap y la aplicación para obtener los datos de NMpa y poder usarlos en la aplicación de la manera más organizada posible}
	{20}
\taskframe
	{2.2.2.4}
	{Enlazado de datos con la GUI}
	{Enlazar los datos con las diferentes vistas a través de diversos controladores, para poder visualizar e interactuar con ellos}
	{15}
\taskframe
	{2.2.3}
	{Testing}
	{Una vez desarrollada la aplicación, realizar un amplio testeo para comprobar que funciona correctamente}
	{10}
\taskframe
	{2.2.4}
	{Corrección de bugs/errores}
	{En base a los errores detectados en el testeo, implementar las correcciones a dichos fallos}
	{15}
\taskframe
	{3.1}
	{Integración del estado del arte}
	{Integrar el estado del arte desarrollado dentro de la memoria}
	{5}
\taskframe
	{3.2}
	{Documentación de la aplicación desarrollada}
	{Elaborar en base a todo el proceso de desarrollo una documentación clara sobre la aplicación e integrarla en la memoria}
	{15}
\taskframe
	{3.3}
	{Resumen de la memoria}
	{Terminar la elaboración de la memoria, añadiendo las diferentes secciones necesarias y el formato correspondiente}
	{5}
\taskframe
	{3.4}
	{Elaborar presentación}
	{Elaborar la presentación en diapositivas que se usará en la defensa ante el tribunal}
	{5}
\taskframe
	{3.5}
	{Preparacion de la defensa}
	{Preparar la defensa ante el tribunal en funcion a la documentación elaborada}
	{10}
\taskframe
	{4}
	{Reuniones periódicas}
	{Reuniones periódicas con el director del TFG para llevar un control del desarrollo del proyecto}
	{15}

\subsection{Entregables}
{\color{red} Entregables durante la Fase 1, si hay (en este caso seria solo la parte de la memoria).}

\subsubsection{Fase 1}
\subsubsection{Fase 2}

\subsection{Cronograma}
{\color{red} Distribución de tareas de la Fase 1 en un cronograma generado con Project o una herramienta similar.}

%------------------------------------------------------------------------------

\section{Fase 2}
{\color{red} Parte de la aplicación.}

\subsection{Tareas}
{\color{red} Todas las tareas de la Fase 2, primero en lista y después como en cuadritos con descripción y tal.}

\subsection{Entregables}
{\color{red} Entregables durante la Fase 2, si hay (aplicación + la otra parte de la memoria que sirve como documentación de la aplicación y así).}

\subsection{Cronograma}
{\color{red} Distribución de tareas de la Fase 2 en un cronograma generado con Project o una herramienta similar.}

%------------------------------------------------------------------------------

\section{Gestión de costos}
{\color{red} Poner los recursos que se necesitan.}

\subsection{Presupuesto}
{\color{red} Todas las tablas y así, todo lo dado en la asignatura de Mari Carmen.}

%------------------------------------------------------------------------------

\section{Gestión de riesgos}

\subsection{Fase 1}
{\color{red} Poner los diferentes riesgos de la Fase 1. En esta fase son pocos o ninguno, por el carácter teórico de la fase.}

\subsubsection{Explicación y plan de contingencia}
{\color{red} Explicar dichos riesgos. Poner por puntos. Puntos como descripción, peligrosidad, y medidas preventivas o para corregir. Nada de párrafos.}

\subsection{Fase 2}
{\color{red} Poner los diferentes riesgos de la Fase 2. En esta fase los risegos son mucho mayores, ya que puede haber desde complicaciones a la hora de programar, limitaciones por no ser root,... Mil historias.}

\subsubsection{Explicación y plan de contingencia}
{\color{red} Explicar dichos riesgos. Poner por puntos. Puntos como descripción, peligrosidad, y medidas preventivas o para corregir. Nada de parrafos.}

