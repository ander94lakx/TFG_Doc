\chapter{Descripción, Objetivos y Motivación del proyecto}

\section{Descripción}

En este trabajo se desarrolla un estudio sobre el campo de la seguridad informática, mas concretamente sobre las diferentes herramientas de seguridad informática. En base a eso, se desarrolla una aplicación para dispositivos móviles que busca, de manera sencilla para el usuario, proporcionar soluciones a tareas recurrentes dentro del campo de la seguridad informática, basandose en herramientas ya existentes. Estas herramientas, usadas por pentesters, analistas forenses o hackers de sobrero blanco permiten elaborar operaciones de todo tipo, desde escanear una red inalámbrica hasta romper el cifrado de un archivo para acceder a la información que contiene.

Gran parte de esta herramientas son gratuitas \cite{ofi-seg-inter} o incluso de software libre \cite{github-sec-showcase}, lo que otorga la posibilidad de que dichas herramientas mejoren continuamente.

Sin embargo, el mayor problema de este tipo de herramientas suelen ser su público objetivo. Normalmente este tipo de herramientas están diseñadas para profesionales del sector, profesionales tanto con conocimientos de seguridad informática como de programación o administración de sistemas. La mayoría de estas herramientas se basan en librerías o frameworks completos, con cierta dificultad de uso, o scripts CLI. Debido a esto, cierta tarea como escanear una red, que para un experto en ciberseguridad o un administrador de sistemas se convierte en 5 segundos tecleando un comando, para un usuario medio se convierte en un auténtico quebradero de cabeza.

\section{Objetivos}

El objetivo de este este proyecto es doble. Por una parte se busca realizar un análisis del campo de la seguridad informática, un \textit{estado del arte} del área que permita vislumbrar cuales son las diferentes aplicaciones de dicho área y a partir de ahí concretar las necesidades más importantes dentro de ese campo para, al final, acabar elaborando una aplicación para Android que nos proporcione ciertas utilidades.

Por otra parte, también se busca que la aplicación a elaborar sirva tanto para usuarios experimentados en la materia como para un público general. Para ello, un buen diseño de la interfaz gráfica (GUI) o diferentes principios de experiencia de usuario (UX) jugarán un papel fundamental. De esta manera lograremos una transición entre herramientas accesibles solo para unos pocos a una herramienta para todo el mundo.

\section{Motivación}

A día de hoy la informática es un industria fundamental dentro de la sociedad en general y de las vidas de las personas en particular. Los ordenadores personales son la herramienta fundamental de trabajo en una gran cantidad de áreas, ademas de una herramienta que se encuentra en prácticamente cualquier hogar. Los denominados Smartphones junto a Internet se han convertido en la principal herramienta de comunicación. La revolución causada por la industria llega hasta tal punto que una organización como la ONU ha declarado Internet \textit{como un derecho humano por ser una herramienta que favorece el crecimiento y el progreso de la sociedad en su conjunto} \cite{onu-internet},.

Teniendo en cuenta toda la información que transmitimos, almacenamos y procesamos, sería lógico pensar que la seguridad de dicha información es vital. El área de la seguridad informática se encarga de ofrecer los mecanismos necesarios para que nuestra información no se vea comprometida de ninguna manera y nuestros dispositivos permanezcan seguros y con la menor cantidad de vulnerabilidades posible. Cada vez este área resulta mas importante, y sobre todo ahora con la llegada del Internet of Things (IoT), ya que pasamos de tener no solo nuestros ordenadores o Smartphones conectados a Internet, sino que tenemos otros dispositivos como nuestro coche o nuestra lavadora conectados, con los riesgos que ello conlleva.