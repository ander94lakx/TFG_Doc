\chapter{Descripción, Objetivos y Motivación}

\section{Descripción}

A lo largo de este Trabajo de Fin de Grado se desarrolla un estudio sobre el campo de la seguridad informática, más concretamente sobre las amenazas y técnicas de seguridad informática actuales. En base a todo ese estudio se desarrolla una aplicación para dispositivos móviles que busca, de manera sencilla para un usuario medio, proporcionar soluciones a tareas recurrentes dentro del campo de la seguridad informática, basándose en herramientas ya existentes. Estas herramientas, usadas por pentesters, analistas forenses o hackers de sombrero blanco permiten elaborar operaciones de todo tipo, desde escanear una red inalámbrica hasta romper el cifrado de un archivo para acceder a la información que contiene.

Gran parte de estas herramientas son gratuitas \cite{ofi-seg-inter} o incluso de software libre \cite{github-sec-showcase}, lo que otorga la posibilidad de que dichas utilidades mejoren continuamente. El mayor problema de este tipo de herramientas suelen ser su público objetivo. Normalmente este tipo de soluciones están diseñadas para profesionales del sector, profesionales tanto con conocimientos de seguridad informática como de programación o administración de sistemas. La mayoría de estas herramientas suelen ser desde grandes librerías o frameworks completos, con cierta dificultad de uso, hasta pequeños scripts CLI (Command Line Interface). Debido a esto, cierta tarea como escanear una red, que para un experto en ciberseguridad o un administrador de sistemas se convierte en 5 segundos tecleando un comando, para un usuario medio se convierte en un auténtico quebradero de cabeza.

\section{Objetivos}

El objetivo de este este proyecto es doble. Por una parte se busca realizar un análisis del campo de la seguridad informática, un \textit{estado del arte} del área que permita vislumbrar cuales son las amenazas existentes, los diferentes campos de enfoque y las técnicas actuales para securizar sistemas para, a partir de ahí, concretar las necesidades más importantes dentro de ese campo. Finalmente, basándose en toda la información recogida se acabará desarrollando una aplicación para Android que nos proporcione ciertas utilidades.

Por otra parte, también se busca que la aplicación a elaborar sirva tanto para usuarios experimentados en la materia como para un público general. Para ello, un buen diseño de la interfaz gráfica (GUI, Graphical User Interface) o diferentes principios de experiencia de usuario (UX, User Experience) jugarán un papel fundamental. De esta manera lograremos una transición entre herramientas accesibles solo para unos pocos a herramientas aptas para el público general.

\section{Motivación}

A día de hoy la informática es una industria fundamental dentro de la sociedad en general y de las vidas de las personas en particular. Los ordenadores personales son la herramienta fundamental de trabajo en una gran cantidad de áreas, además de una herramienta que se encuentra en prácticamente cualquier hogar. Los denominados Smartphones, en conjunción a Internet, se han convertido en la principal herramienta de comunicación. La revolución causada por la industria llega hasta tal punto que una organización de la relevancia de la ONU declara y define Internet como un derecho humano \textit{por ser una herramienta que favorece el crecimiento y el progreso de la sociedad en su conjunto} \cite{onu-internet}.

Teniendo en cuenta toda la información que transmitimos, almacenamos y procesamos, sería lógico pensar que la seguridad de dicha información es vital. El área de la seguridad informática se encarga de ofrecer los mecanismos necesarios para que nuestra información no se vea comprometida y nuestros dispositivos permanezcan seguros y con la menor cantidad de vulnerabilidades posible. Cada vez resulta mas importante, y sobre todo ahora con la llegada del Internet of Things (IoT), ya que pasamos de tener, no solo nuestros ordenadores o Smartphones conectados a Internet, a dotar de conexión a Internet a otros dispositivos como nuestro coche o nuestra lavadora, con los riesgos que ello conlleva.

Cada vez el área de la seguridad informática esta creciendo más, pero el enfoque de la seguridad informática no puede ser unilateral. Para garantizar en la mayor medida posible la seguridad de los sistemas, el usuario debe tomar un papel activo. Si los usuarios disponen de herramientas a us alcance para llevar a cabo tareas que permitan que sus sistemas estén más seguros, jugarán un papel activo, y no pasivo, dentro de la seguridad de su información. Dotar al usuario de la capacidad, que no de la necesidad, de hacer más seguros sus sistemas sirve para complementar el papel de los expertos y profesionales de la seguridad informática.