\chapter{Introducción}

\epigraph{\textit{''La mejor forma de predecir el futuro es implementarlo''}}{--- David Heinemeier Hansson}

Como se ha comentado a lo largo de este informe, y tras el posterior estudio sobre el área de la seguridad informática, el objetivo de este TFG radica en desarrollar una aplicación con diversas herramientas de escaneo de redes y recogida de información, que puedan ser útiles para auditores y expertos en seguridad informática. A su vez, se tiene como objetivo acercar este tipo de herramientas a un público más general, usando una interfaz gráfica sencilla, intuitiva y agradable junto con diferentes elementos que mejoren la experiencia de usuario.

Durante esta parte del informe se documentará todo el proceso de desarrollo de la aplicación, empezando por enumerar las herramientas que se usarán durante dicho proceso.

También se hará hincapié en la GUI (Graphical User interface) generada, de la misma manera que en principios de UX (User eXperience), elementos que quedarán reflejados en este informe. Tras ello se documentará el proceso de desarrollo, testeo y depuración, donde se profundizará en como se desarrollan los diferentes elementos que a nivel técnico requieren especial atención.

Para el final de esta parte se obtendrá una aplicación funcional, depurada y que satisfaga los Requisitos Funcionales establecidos al inicio.

