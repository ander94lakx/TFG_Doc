\chapter{Testeo y corrección de errores}

\epigraph{\textit{''El testing de componentes puede ser muy efectivo para mostrar la presencia de errores, pero absolutamente inadecuado para demostrar su ausencia''}}{--- Edsger Dijkstra}

{\color{red} Tras programarlo todo pueden salir fallos, hay que hacer pruebas, e incluso habrá que cambiar cosas o añadir cosas nuevas. Todo eso va explicado aquí. NO CONFUNDIR con añadir nuevos requisitos funcionales o que se haya alargado el tiempo, eso va en la última parte, en la de conclusiones}