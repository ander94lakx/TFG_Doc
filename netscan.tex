\chapter{Código fuente y documentación del proyecto}

En todo momento se ha desarrollado este proyecto para que sirviera como base para cualquier usuario para entender aspectos sobre Seguridad Informática, además de para mostrar cómo dotar de soluciones a problemas concretos. Por otro lado, la aplicación obtenida se ha desarrollado teniendo en cuenta que fuera lo más escalable posible, tanto a nivel de funcionalidad, como en el apartado gráfico. La aplicación pretende una solución a un problema concreto, pero también pretende ser extensible a otros ámbitos, además de dar la capacidad de poder ser mejorada.

Se pretende que la aplicación sea accesible por cualquiera. Debido a esto, la aplicación está disponible como software libre, bajo una licencia abierta, que permite que cualquier usuario pueda tanto analizar el código y entender cómo funciona como disponer del código para, en base a él, desarrollar sus propias soluciones.

Existen muchos motivos para apostar por el software libre. Está claro que todo el desarrollo de la informática sería completamente diferente si todo el software del que disponemos fuera privativo. El software libre ha sido una rama fundamental a la hora de que se desarrollen los sistemas y las tecnologías de la información. A nivel tecnológico en general, y informático en particular, viviríamos en un mundo mucho menos desarrollado tecnológicamente si no fuera por todos esos programadores, desarrolladores, ingenieros, científicos y organizaciones sin ánimo de lucro que desarrollan tanto software libre como conocimiento abierto. Software y conocimiento que permite que la informática sea accesible a todos, sin que esté en manos de unos pocos adinerados.

Tal y como dice Richard Stallman\footnote{\url{https://www.20minutos.es/noticia/199929/0/Stallman/Premio/softwarelibre/}},  fundador del proyecto GNU y de la Free Software Foundation, \textit{las obras de conocimiento deben ser libres}. Todo lo aprendido durante este proyecto y la aplicación desarrollada puede ser de gran ayuda para usuarios y desarrolladores que simplemente quieran aprender o quieran basarse en este trabajo para sus propios desarrollos. Por ello, tanto este mismo informe como el código íntegro de la aplicación desarrollada serán liberados, una vez presentado este Trabajo de Fin de Grado, con licencias abiertas.

En el caso del documento, y como ya se añade en el propio documento, se liberará mediante la GNU Free Documentation License. En el caso del código se liberará mediante la licencia GPLv3.

El código de la aplicación se encuentra disponible en es siguiente enlace:

\url{https://github.com/ander94lakx/NetScan}

Esta misma memoria se encuentra disponible en el siguiente enlace: 

\url{https://github.com/ander94lakx/TFG_Doc}

